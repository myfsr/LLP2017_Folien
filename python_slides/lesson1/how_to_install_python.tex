\documentclass{beamer}

\usetheme{Boadilla}

\usepackage[utf8]{inputenc}
\usepackage{listings}
\usepackage{color}

\definecolor{mygreen}{rgb}{0,0.6,0}
\definecolor{mygray}{rgb}{0.5,0.5,0.5}
\definecolor{mymauve}{rgb}{0.58,0,0.82}

\lstset{
    backgroundcolor=\color{white},
    % choose the background color;
    % you must add \usepackage{color} or \usepackage{xcolor}
    basicstyle=\footnotesize\ttfamily,
    % the size of the fonts that are used for the code
    breakatwhitespace=false,
    % sets if automatic breaks should only happen at whitespace
    breaklines=true,                 % sets automatic line breaking
    captionpos=b,                    % sets the caption-position to bottom
    commentstyle=\color{mygreen},    % comment style
    % deletekeywords={...},
    % if you want to delete keywords from the given language
    extendedchars=true,
    % lets you use non-ASCII characters;
    % for 8-bits encodings only, does not work with UTF-8
    frame=single,                    % adds a frame around the code
    keepspaces=true,
    % keeps spaces in text,
    % useful for keeping indentation of code
    % (possibly needs columns=flexible)
    keywordstyle=\color{blue},       % keyword style
    % morekeywords={*,...},
    % if you want to add more keywords to the set
    numbers=left,
    % where to put the line-numbers; possible values are (none, left, right)
    numbersep=5pt,
    % how far the line-numbers are from the code
    numberstyle=\tiny\color{mygray},
    % the style that is used for the line-numbers
    rulecolor=\color{black},
    % if not set, the frame-color may be changed on line-breaks
    % within not-black text (e.g. comments (green here))
    stepnumber=1,
    % the step between two line-numbers.
    % If it's 1, each line will be numbered
    stringstyle=\color{mymauve},     % string literal style
    tabsize=4,                       % sets default tabsize to 4 spaces
    % show the filename of files included with \lstinputlisting;
    % also try caption instead of title
    language = Python,
	showspaces = false,
	showtabs = false,
	showstringspaces = false,
	escapechar = ,
}


\title{Grundlagen und Grundideen für Python}
% \subtitle
\author{Felix V.}
\date{\today}

%\lstset{
%  backgroundcolor=\color{black},
%  basicstyle=\color{green}
%}


%%%%%%%%%%%%%%%%%%%%%%%%%%%%%%%%%%%%%%%%%%%%%%%%%%%%%%%%%%%%%%%%%%%%%%%%%

\begin{document}
\begin{frame}
\titlepage
\end{frame}

\begin{frame}
  \frametitle{Grundlagen und Grundideen für Python}
  \tableofcontents
\end{frame}

\begin{frame}
  \frametitle{Software}
  \begin{itemize}
  \item<+-> Interpreter
    \begin{itemize}
      \item python3.4-3.7
    \end{itemize}
  \item<+-> Editoren
    \begin{itemize}
    \item Emacs
    \item VIM
    \item Atom
    \item Sublimetext3
    \item IDLE
    \end{itemize}
  \item<+-> IDEs
    \begin{itemize}
    \item pycharm (Jetbrains)
    \item Eric
    \item Spider3
    \end{itemize}
  \item<+-> Package Manager
    \begin{itemize}
    \item pip
    \end{itemize}
  \item<+-> iPython
  \end{itemize}
\end{frame}

\section{Interpreter VS Compiler}
\begin{frame}
  \frametitle{Interpreter VS Compiler}
  \begin{itemize}
  \item<+-> Compiler
    \begin{itemize}
    \item übersetzt in Maschinensprache
    \item erstellt eigene Datei (executable)
    \end{itemize}
  \item<+-> Interpreter
    \begin{itemize}
    \item erstellt keine eigene Datei
    \item liest eine Zeile und führt sie aus (oft langsamer als Compiler)
    \item Ermöglicht die einfache Umsetzung eines REPLs
    \end{itemize}
  \end{itemize}
\end{frame}

\section{Shell und REPL}
\begin{frame}[fragile]
  \frametitle{Shell und REPL}
  \begin{itemize}
    \item<1-> die Prompt beginnt mit \lstinline+>>>+
    \item<1-> man kann validen Python code schreiben und mit $<$Enter$>$
    \item<1-> öffnen entweder iPython oder IDLE starten oder in eurem Terminal python3 ausführen
    \item<2-> Aufgabe 1:\begin{itemize}
      \item gebt euren ersten Python Befehl ein, ein kleines Hello Python \\
        \begin{lstlisting}
          >>> print('Hello Python')
          \end{lstlisting}
    \end{itemize}
    \item<3-> man kann alles von der Python shell aus ausführen
  \end{itemize}
\end{frame}


\section{Typen - naja fast}
\begin{frame}
\frametitle{Typen - naja fast}
Python ist eine schwach typisierte Scriptsprache (weakly typed scripting language). Es gibt Typen (anders als in JavaScript), aber Variablen haben keine festen Typen.
\end{frame}

\begin{frame}
  \frametitle{Typen - naja fast}
	\textbf{builtin Datentypen:}\\
	\begin{tabular}{c|l}
		Name & Funktion \\ \hline
		\texttt{object} & Basistyp, alles erbt von \texttt{object} \\
		\texttt{int} & Ganzzahl "beliebiger" Größe \\
		\texttt{float} & Kommazahl "beliebiger" Größe \\
		\texttt{bool} & Wahrheitswert (\texttt{True}, \texttt{False})\\
		\texttt{None} & Typ des \texttt{None}-Objektes \\
		\texttt{type} & Grundtyp aller Typen (z.B. \texttt{int} ist eine Instanz von \texttt{int}) \\
		\texttt{list} & standard Liste \\
		\texttt{tuple} & unveränderbares n-Tupel \\
		\texttt{set} & (mathematische) Menge von Objekten \\
		\texttt{frozenset} & unveränderbare (mathematische) Menge von Objekten \\
		\texttt{dict} & Hash-Map \\
	\end{tabular}
\end{frame}

\section{Operationen, ein praktischer Taschenrechner}
%\begin{frame}
%
% \begin{itemize}
%  \item int ist nicht beschränkt
%  \item Es gibt \lstinline?+,-,*,/,**?
%    \begin{itemize}
%    \item
%    \textcolor{red}{Vorsicht} unterschied zwischen Python 2.7 und 3.*:\\
%      \lstinline+/+ ist in Python 3 keine Integer Division
%       \end{itemize}
%  \end{itemize}
%\end{frame}

\section{Operatoren}
\begin{frame}[fragile]{Operatoren}
   \frametitle{Python als praktischer Taschenrechner} 
   \begin{description}
       \item<+-> int ist nicht beschränkt
	\item<+->[mathematisch] \alert{\texttt{+}}, \alert{\texttt{-}}, \alert{\texttt{*}}, \alert{\texttt{/}} \\
             \textcolor{red}{!Vorsicht!} unterschied zwischen Python 2.7 und 3.*:\\
      \lstinline+/+ ist in Python 3 keine Integer Division
	    \item<+->[vergleichend] \alert{\texttt{<}}, \alert{\texttt{>}}, \alert{\texttt{<=}}, \alert{\texttt{>=}}, \alert{\texttt{==}} (Wert gleich), \alert{\texttt{is}} (gleiches Objekt/gleiche Referenz)
	    \item<+->[logisch] \alert{\texttt{and}}, \alert{\texttt{or}}, \alert{\texttt{not}}\alert{\texttt{(a \&\& b) || (!c)}} aus C oder Java entspricht \alert{\texttt{(a and b) or not c}} in Python
	    \item<+->[bitweise] \alert{\texttt{\&}}, \alert{\texttt{|}}, \alert{\texttt{<<}}, \alert{\texttt{>>}}, \alert{\texttt{\^}} (xor), \alert{\texttt{\~}} (invertieren)
	    \item<+->[Accessoren] \alert{\texttt{.}} (für Methoden und Attribute), \alert{\texttt{[]}} (für Datenstrukturen mit Index)
	\end{description}
\end{frame}


\section{Skripte}
\begin{frame}[fragile]
  \frametitle{Skripte}
  \begin{itemize}
  \item<1-> scripte enden mit *.py
  \item<1-> scripte sind in utf-8
  \item<2-> scripte sollten mit einem schbang/hashbang beginnen
    \begin{lstlisting}
      #!/usr/bin/env python3
    \end{lstlisting}
  \item<3-> zu pfaden, ordnern zu libraries und main functionen kommen wir später
  \end{itemize}
\end{frame}

\end{document}

%%% Local Variables:
%%% mode: latex
%%% coding: utf-8
%%% TeX-master: t
%%% End:
%vim:fdm=marker set ft=tex:
