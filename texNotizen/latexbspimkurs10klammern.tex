% Dokumentenklasse
% immer am Anfang des Dokuments
\documentclass{scrartcl}

% Präambel
\usepackage[utf8]{inputenc} % Eingabekodierung
\usepackage[T1]{fontenc}    % Schrifkodierung
\usepackage[ngerman]{babel} % optimiert für Deutsch (z.B. Titel, Silbentrennung)

\usepackage{amsmath}        % (nahezu) unerlässliches Paket für den Mathemodus
\usepackage{amssymb}        % ebenso
\title{Mathematikmodus}
\author{LLP-Kurs}
\date{nach Ostern}

\newcommand{\setDef}[2]{\left\{#1 
  \,\middle\vert\, #2\right\}}

% Ende Präambel
% Beginn eigentliches Dokument
\begin{document}
\maketitle
  Klammmern können zu groß werden:
  \begin{equation}
    \left(\sum_{i=0, \dots 5} 5 = 25\right)
  \end{equation}

  Sei $A = \{a \in B \mid \lnot a \equiv b\}$ eine Menge, bei der man mid nutzen kann.
  \begin{align}
    C = \left\{a^{\sum_{i = 1}^n i} \middle| n = m_7\right\} \\
    C = \left\{a^{\sum_{i = 1}^n i} \mid n = m_7\right\}
  \end{align}
  $C$ ist keine solche Menge. Der Kompromiss kann ein eigenes Makro sein:
  \begin{equation}
    C = \setDef{a^{\sum_{i = 1}^n i}}{n = m_7}
  \end{equation}
  Übersichtlicher ist der Code auch.
\end{document}
