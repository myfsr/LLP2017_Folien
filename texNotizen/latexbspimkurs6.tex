% Dokumentenklasse
% immer am Anfang des Dokuments
\documentclass{scrartcl}

% Präambel
\usepackage[utf8]{inputenc} % Eingabekodierung
\usepackage[T1]{fontenc}    % Schrifkodierung
\usepackage[ngerman]{babel} % optimiert für Deutsch (z.B. Titel, Silbentrennung)

\title{Mein erstes \LaTeX-Dokument}
\author{LLP-Kurs}
\date{nach Ostern}

\newcommand{\wichtig}[1]{(Wichtig!) \emph{#1} (Achtung, dieses Wort sollte man sich merken!)}
\newcommand{\tb}{\textbackslash}

\renewcommand{\maketitle}{
  Titel: Ein \LaTeX-Dokument

  geschrieben für den LLP-Kurs

  Sommersemester 2017

}

% Ende Präambel
% Beginn eigentliches Dokument
\begin{document}
\maketitle
  Das eigentliche Dokument         wird      nun
  zwischen
  \tb begin\{document\} und
  \tb end  \{document\}
  reingeschrieben.
  \TeX \wichtig {formatiert}        es dann \wichtig {ansprechend}.

  Die Formatierung des Quelltextes ist (wie gesagt) (fast) beliebig.  Diese
  Formatierung wird allerdings nicht unbedingt im Dokument wieder gespiegelt.

  Diese Freiheit kann man nutzen um
    den Quelltext übersichtlich zu gestalten,
    zusammenhängende Blöcke wie beim Programmieren           einzurücken,
    Tabellen in      ASCII-Art zu erstellen, also aneinander auszurichten,
    strukturierende Makros in eigene Zeilen zu schreiben
   und
    überlange Zeilen zu vermeiden.
  %
  Der vorhergehende Satz ist im fertigen Dokument hintereinandergereiht wie ein üblicher langer Satz.
  %
  % Leerzeilen können der Übersichtlichkeit helfen.
  % und wenn ein Kommentar sind erzeugen sie keinen neuen Absatz.

  Blöcke sind beispielsweise alles, was zwischen einem Paar
  \tb begin\{etwas\}
    und
  \tb end  s\{etwas\}
  steht.

  \tb begin und \tb end sind solche strukturierenden Makros.

  Ich kann auch Makros ineinander tun: Jetzt kommt ein echt
  wichtiges Wort:\par
  \emph{
    \wichtig {Inhalte von Makros einzurücken, kann hilfreich beim Lesen sein}
  }
\end{document}
