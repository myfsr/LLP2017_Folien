\documentclass{latex-htw-uebung}

\title{1. Übungsblatt}
\date{9.\ November 2015}

\begin{document}

Hier ist ein erstes, kleines \LaTeX-Programm:

\begin{lstlisting}
\documentclass[ngerman]{scrartcl}

\usepackage[T1]{fontenc}
\usepackage[utf8]{inputenc}
\usepackage{babel}

\title{Mein erstes \LaTeX-Dokument}
\author{Das ist von mir!}
\date{Stardate 47943.2}

\begin{document}

\maketitle

Das ist ja einfach!

\end{document}
\end{lstlisting}

Versuche zuerst, dieses Programm in einen \LaTeX-Editor einzugeben und es zu übersetzen.

Sobald das geklappt hat, spiele ein wenig mit dem Code herum:
\begin{itemize}
\item Was geschieht, wenn Du statt \lstinline{scrartcl} die \emph{Dokumentenklasse}
  \lstinline{scrbook} verwendest?  Was geschieht bei \lstinline|scrreprt|,
  \lstinline|article|, \lstinline|book|, \lstinline|report|?
\item Was passiert, wenn man vor \lstinline|\begin{document}| einfach das Wort
    \lstinline|Hallo| schreibt?  Was geschieht, wenn man es nach
    \lstinline|\end{document}| schreibt?
\item Versuche, so viel Text an den Satz \lstinline|Das ist ja einfach!| anzufügen, dass
  mehrere Zeilen im erstellten \LaTeX-Dokument entstehen.  Welche Form hat der Absatz?
  (Blocksatz, Flattersatz, \dots) Gibt es Trennungen?
\item Schreibe direkt vor \lstinline|Das ist ja einfach!| den Code
  \lstinline|\begin{center}|.  Schreibe hinter dem Absatz auf eine
    neue Zeile den Code
    \lstinline|\end{center}|.  Was geschieht?
\item Schreibe vor die letzte Zeile folgenden Code
\begin{lstlisting}
\section{Hier geht's los}

Ein Absatz beginnt jeweils mit einer neuen Zeile.  Wenn man also noch
einen Absatz haben möchte, lässt man einfach etwas Platz.

Etwa so.
\end{lstlisting}
  Was geschieht?
\item Wie bekommt man einen Unterabschnitt?  Wie einen Unterunterabschnitt?
\item Füge irgendwo vor \lstinline|\end{document}| das Kommando
\lstinline|\diesenBefehlGibtEsGarNicht| ein.  Wie reagiert \LaTeX?
\item Füge in der \emph{Präamble}, also nach der ersten Zeile, aber noch vor
  \lstinline|\begin{document}| die Zeile \lstinline|\usepackage{blindtext}| ein.  Verwende
    dann den Befehl \lstinline|\blindtext| im Textteil des Dokuments.
    Was geschieht?
\item Ersetze \lstinline|\date{Stardate 47943.2}| durch \lstinline|\date{\today}|. Wie
  ändert sich das Datum? Was steht da wohl morgen?
\end{itemize}

\end{document}

%%% Local Variables:
%%% mode: latex
%%% TeX-master: t
%%% ispell-local-dictionary: "de_DE"
%%% End:
