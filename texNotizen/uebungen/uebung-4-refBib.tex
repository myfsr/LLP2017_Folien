\documentclass{latex-htw-uebung}

\title{4.~Übungsblatt}
\date{23.\ Mai~2017}

\begin{document}

\NewTask Erstelle ein Dokument mit einem Abschnitt, auf den du mit \lstinline|\label{sec:inhalt}| und \lstinline|\ref{sec:inhalt}| verweist. Übersetze das Dokument \emph{einmal} mit \lstinline|pdflatex|. (Vermutlich musst du es über ein Terminal machen, da Editoren oft automatisch mehrmals übersetzen. Lösche vorher alle \lstinline|.aux|-Dateien.) Ist der Verweis richtig eingefügt? Schaue dir die entstandene \lstinline|.aux|-Datei an. Erkennst du, welche Zeile zum Label gehört?
\NewTask Gehe zur Website der SLUB (\url{http://slub-dresden.de}), suche da nach
einem von Dir kürzlich gelesenen Buch, und schaue dir den von der Website dazu
bereitgestellten \textsc{Bib\TeX}-Eintrag an.

\NewTask Lege eine Datei mit dem Namen \texttt{meineBuecher.bib} an und füge
folgenden Inhalt ein:

\begin{lstlisting}
@book{MP1,
  title  = {Wie ich \LaTeX{} meisterte},
  author = {Max Power},
  editor = {Santos L. Halper},
  address = {Springfield},
  publisher = {Läufer Verlag},
  year   = {1999},
  isbn   = {3-1234-4321-3},
}
@article{wichtigerArtikel,
  author = {Maxine Power and Roy Force and Chesty LaRue },
  title = {Die neuesten Entwicklungen in guter Typographie},
  journal = {Neue Typographie},
  volume = {23},
  number = {3},
  year = {2013},
  pages   = {42-84},
}
\end{lstlisting}
Füge noch einen weiteren \textsc{Bib\TeX} Eintrag vom Typ
\lstinline|@article| mit dem Schlüsselwort \lstinline|wa2| hinzu, wobei
alle Daten deiner Fantasie überlassen sind.


Erstelle weiter eine Datei \texttt{meineAusarbeitung.tex} mit
folgenden Inhalt:

\begin{lstlisting}
\documentclass[a4paper,ngerman]{article}

\usepackage{babel}
\usepackage[utf8]{inputenc}
\usepackage[T1]{fontenc}
\usepackage[backend=bibtex,
            style=numeric-comp, backref=false,
            (*@autocite@*)=footnote, maxnames=2,
            isbn=true]{biblatex}
\addbibresource{meineBuecher.bib}
\title{Mein Werk}
\author{Mein Name}

\begin{document}
\maketitle
Motiviert durch das \LaTeX{}-Standardwerk~\cite{MP1} habe ich meine Studien zu
diesem Thema vertieft. Der aktuelle Stand der Forschung wird
in~\cite{wichtigerArtikel} wiedergegeben. Auch wichtig in diesem Zusammenhang
ist~\cite{wa2}.

\printbibliography
\end{document}
\end{lstlisting}
Führe nun nacheinander \texttt{pdflatex meineAusarbeitung.tex},
\texttt{bibtex meineAusarbeitung}, \texttt{pdflatex
  meineAusarbeitung.tex}, \texttt{pdflatex meineAusarbeitung.tex}
aus. Schau Dir das resultierende Dokument an und führe die Übersetzungsschritte nach jeweils jeder der
folgenden Veränderungen wieder durch und betrachte das Ergebnis.

\begin{itemize}

\item Ersetze \lstinline|maxnames=2| durch \lstinline|maxnames=3|.
\item Ersetze \lstinline|style=numeric-comp| durch
  \lstinline|style=alphabetic|.
\item Ersetze ein  \lstinline|\cite| durch \lstinline|\footcite|.
\item Ersetze das \lstinline|\footcite| durch \lstinline|\autocite|.
\item Ersetze \verb|autocite=footnote| durch \verb|autocite=inline|,
\item Ersetze \lstinline|ngerman| durch \lstinline|american|.
\item Ersetze \lstinline|backref=false| durch
  \lstinline|backref=true|.
\end{itemize}

\end{document}

%%% Local Variables:
%%% mode: latex
%%% TeX-master: t
%%% ispell-local-dictionary: "de_DE"
%%% End:
