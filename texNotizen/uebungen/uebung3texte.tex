Differentialgleichungen

Typen
-----

Wenn man Differentialgleichungen betrachtet, teilt man sie in verschiedene Typen auf, die mit verschiedenen Ansätzen gelöst werden können.

Gewöhnlichen Differentialgleichung
explizite ODE
lineare ODE
partielle Differentialgleichung

Dabei ist Rmm der Raum der quadratischen Matrizen der Größe m. Ein Element ist die Vandermonde-Matrix mit der folgenden Eigenschaft.

% Die Vandermonde-Determinante

Bemerkung
---------
Oftmals werden physikalische Zusammenhänge vereinfacht bevor man sie mit mathematischen Methoden bearbeitet. Das Pendel kann durch die Gleichungen beschrieben werden.

% Bild von einem Pendel finden und einfügen

(nicht-lineare Differentialgleichung 2. Ordnung)

Aufgrund der Annährung sin α ≈ α vereinfacht sich dies zu

Analytisches Handwerkszeug
--------------------------
Stirlingformel
Die Stirlingformel ist die Approximation für n ∈ ℕ

Satz
Der nächste Term in der Fehlerapproximation O(ln(n)) in der Stirlingformel ist , sodass sich 

ergibt. Tatsächlich hat die Stirlingformel als Approximationsformel für die Fakultätsfunktion die Eigenschaft, dass

Beweis siehe https://en.wikipedia.org/wiki/Stirling%27s_approximation

Weiteres analytisches Handwerkszeug beinhaltet
Faltung -- das mit *
mehrdimensionale Integration -- das mit
