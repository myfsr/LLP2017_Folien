% Dokumentenklasse
% immer am Anfang des Dokuments
\documentclass{scrartcl}

% Präambel
\usepackage[utf8]{inputenc} % Eingabekodierung
\usepackage[T1]{fontenc}    % Schrifkodierung
\usepackage[ngerman]{babel} % optimiert für Deutsch (z.B. Titel, Silbentrennung)

\usepackage{amsmath}        % (nahezu) unerlässliches Paket für den Mathemodus
\usepackage{amssymb}        % ebenso
\title{Mathematikmodus}
\author{LLP-Kurs}
\date{nach Ostern}

% Ende Präambel
% Beginn eigentliches Dokument
\begin{document}
\maketitle
  Was passiert mit Absätzen im Mathemodus?
  \begin{align}
    a^b = \frac {1}{d+c}

    7 = 7
  \end{align}

  Manchmal möchte man Schritte begründen:
  \begin{equation}
    \cos \alpha + \beta = \cos \alph \sin \beta + \sin \alpha \cos \beta (Additionstheorem)
  \end{equation}
\end{document}
