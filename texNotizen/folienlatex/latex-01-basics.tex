\errorcontextlines=5

%%%

\documentclass[
%handout,
]{beamer}

\usepackage{ifluatex}
\ifluatex\else\errmessage{This document requires LuaLaTeX}\fi

\usepackage{etex,etoolbox}
\usepackage{fontspec}
\usepackage[ngerman]{babel}
\usepackage{csquotes}
\usepackage{array}
\usepackage{wrapfig}
\usepackage{booktabs}
\usepackage{ccicons}
\usepackage{calc}

\usepackage{tikz}
\usetikzlibrary{arrows,intersections,calc,through,%
  external,positioning,automata,datavisualization,%
  datavisualization.formats.functions}

\usepackage{luacode}
\usepackage{pgfplots}
\usepackage{manfnt}

%%% title and such

\title{Wissenschaftliches Arbeiten mit \LaTeX}
\author{Felix Hilsky\\basierend auf einem Kurs von Daniel Borchmann und Tom Hanika}
\titlegraphic{\ccLogo \ccAttribution \ccShareAlike}

%%% theme

\usepackage{tikz}
\usetikzlibrary{shapes.multipart}
\usetheme{CambridgeUS}
\setbeamertemplate{blocks}[rounded][shadow=false]
\setbeamertemplate{items}{\raisebox{0.3ex}{%
    \tikz[scale=0.13] \draw[fill] (0,0) -- (0,1) -- (0.9,0.5) -- cycle;}}
\usetikzlibrary{arrows}
\tikzset{>={stealth'[sep]}}
\setbeamertemplate{navigation symbols}{}
\setbeamertemplate{footline}{}
\setlength{\abovedisplayskip}{0pt}
\setbeamerfont{title}{series=\bfseries}
\defbeamertemplate{block alerted begin}{bends}{%
  \begin{columns}
    \begin{column}{0.05\linewidth}
      \dbend
    \end{column}
    \begin{column}{0.95\linewidth}
      \vskip.75ex\usebeamercolor[fg]{block title
        alerted}\insertblocktitle{}
      \vskip.1em
      \usebeamercolor[fg]{normal text}
}
\defbeamertemplate{block alerted end}{bends}{%
    \end{column}
  \end{columns}
}
%%%

\mode<handout>{
  \usepackage{pgfpages}
  \pgfpagesuselayout{2 on 1}[a4paper,border shrink=5mm]
}

%%% lecture organization

\usepackage{xparse}
\DeclareDocumentCommand \Lecture { m m }{%
  \lecture{#1}{#2}
  \part{#1}
  \include{#2}
}

\AtBeginSection{
  \setbeamertemplate{blocks}[rounded][shadow=true]
  \begin{frame}[plain]
    \begin{block}{}
      \begin{center}
        \textcolor{darkred}{\textbf{\Large \strut\smash{\insertpart}}}\\[1ex]
        \textcolor{blue!70!black}{\strut\smash{\insertsection}}
      \end{center}
    \end{block}
  \end{frame}
  \setbeamertemplate{blocks}[rounded][shadow=false]
  \setbeamertemplate{block alerted begin}[bends]
  \setbeamertemplate{block alerted end}[bends]
}

%%% misc

\newcommand{\GNULinux}{GNU\lower-0.25ex\hbox{/}Linux}
\newcommand{\TikZ}{Ti\emph{k}Z}

\usepackage{listings}

\lstset{language=[LaTeX]TeX, basicstyle=\ttfamily,
  keywordstyle={\color{blue}\bfseries}, frame=tb, extendedchars=true, literate=%
  {ä}{{\"a}}1 {ö}{{\"o}}1, escapeinside={(*@}{@*)}, mathescape=true,
  basewidth=0.5em, keywordstyle={\color{blue}},
  morekeywords={[0]includegraphics,rotatebox,scalebox,resizebox,providecommand,
    subsection,subsubsection,paragraph,subparagraph,part,chapter,tableofcontents,
    mathring,text,mathbb,printindex,addbibresource,printbibliography,subtitle,
    institute,titlegraphic,subject,keywords,draw,path,color,textcolor,toprule,
    midrule,bottomrule,maketitle,setlength,enquote,listoffigures,listoftables,
    theoremstyle,theoremheaderfont,theorembodyfont,newblock,parencite,footcite,
    autocite,bibitem,middle,tikzset,usetikzlibrary,coordinate,node,foreach,
    datavisualization,varepsilon,autocite,bibitem,DeclareRobustCommand,
    DeclareDocumentCommand,IfBooleanTF,bye,frametitle,setbeamertemplate,pause,
    onslide,uncover,visible,invisible,only,alt,temporal,alert,AtBeginSection,
    usetheme,setbeamerfont,tikz,includeonlyframes,mode,pgfpagesuselayout,RequirePackage,
  },
}

\AtBeginDocument{\frame[plain]{\maketitle}}

%%% end of preamble
\subtitle{Einführung}
\date{Sommersemester 2017}

\begin{document}
\section{Aufbau eines \LaTeX-Dokuments}

\begin{frame}[fragile]
  \frametitle{Standard-Dokumentenklassen}
  \begin{itemize}
    \item Spezifiziert das allgemeine Aussehen des Dokuments (Artikel, Report, Buch,
    Brief, \dots)
    \item Wird (im allgemeinen) als erstes im Dokument angegeben
  \end{itemize}  
  \begin{description}
    \item[article] Standardklasse zum Erstellen von einfachen Dokumenten
    \item[report] Standardklassen zum Erstellen längerer Arbeiten
    \item[book] Standardklassen zum Erstellen von Büchern
    \item[scrartcl, scrreprt, scrbook] KOMA-Script Varianten von article, report, book mit
      europäischen Standardwerten
    \item[memoir] Individuell anpassbare Dokumentenklasse
    \item[minimal] Minimale Dokumentenklasse
  \end{description}
  \begin{itemize}
    \item<+-> Können Optionen bekommen
    \begin{lstlisting}
      \documentclass[a4paper,english,draft]{article}
    \end{lstlisting}
  \end{itemize}
\end{frame}

\begin{frame}[fragile]
  \frametitle{Die Präambel}
  \onslide<1->

  \begin{itemize}
  \item<1-> Wird verwendet, um
    \begin{itemize}
    \item<2-> Pakete einzubinden
    \item<3-> Standardwerte des Dokuments anzupassen
    \item<4-> separate Befehle (\emph{Makros}) zu definieren oder zu ändern
    \end{itemize}
  \item<2-> Pakete werden eingebunden mittels
\begin{lstlisting}
\usepackage[(*@\textit{option}@*)]{(*@\textit{paketname}@*)}
\end{lstlisting}
  \end{itemize}

\end{frame}

\begin{frame}[fragile]
  \frametitle{Der \enquote{Dokumentenkörper}}

  Das eigentliche Dokument wird nun zwischen \lstinline!\begin{document}! und
    \lstinline!\end{document}! angegeben.  Dabei kann der Text \enquote{fast} beliebig
  eingegeben werden.  Dieser wird dann von \TeX\ ent- und ansprechend formatiert.

\begin{lstlisting}
\begin{document}
Bei Fülltexten sollte man drauf achten, dass es nicht
allzu viel Sinn macht, lange darüber nachzudenken, was
man wie schreibt.
\end{document}
\end{lstlisting}

  wird zu

%   Makros steuern die Formatierung des Textes in \LaTeX:

%   \begin{itemize}
%   \item<+-> sie beginnen mit \textbackslash, gefolgt von einer Folge von Buchstaben (keine
%     Zahlen!)
%   \item<+-> Mögliche Formen von Argumenten sind
%     \begin{itemize}
%     \item \lstinline!{$\textit{Argument}$}! bezeichnet \emph{obligatorische} Argumente
%     \item \lstinline![$\textit{Argument}$]! bezeichnet \emph{optionale} Argumente
%     \item \lstinline!($\textit{x},\textit{y}$)! oder \lstinline!<$\dots$>! sind auch
%       üblich
%     \end{itemize}

%     Aber das ist \enquote{nur} Konvention!
%   \end{itemize}
% \end{frame}

% \begin{frame}[fragile]
%   \frametitle{Eigene Makros}
%   \onslide<+->

%   Eigene Makros ermöglichen komplexe Formatierungen und verringern Schreibaufwand.  In
%   \LaTeX\ werden eigene Makros durch
% \begin{lstlisting}
% \newcommand{$\textit{Name}$}[$\textit{Anzahl\_der\_Argumente}$]{$\textit{Text}$}
% \end{lstlisting}
%   definiert. Dabei wird in Text jedes Vorkommen von \#\emph{Nummer} durch den Wert des
%   \emph{Nummer}-ten Arguments \emph{syntaktisch} ersetzt.

%   \onslide<+->

%   \begin{Beispiel}
% \begin{lstlisting}
% \newcommand{\blue}[1]{\textcolor{blue}{#1}}
% \end{lstlisting}
%     und der Aufruf \lstinline!\blue{Mein Text}! erzeugt \textcolor{blue}{Mein Text}.
%   \end{Beispiel}
% \end{frame}

% \begin{frame}[fragile]
%   \frametitle{Weitere Makro-Definitions-Befehle}

%   \onslide<+->

%   \begin{itemize}[<+->]
%   \item Sollte es ein Makro schon geben, so redefiniert \lstinline!\renewcommand!
%     dieses Makro
%   \item \lstinline!\providecommand! definiert ein Makro nur, wenn es vorher noch nicht
%     existierte
%   \item Die Varianten \lstinline!\newcommand*!, \lstinline!\renewcommand*! bzw.\,
%     \lstinline!\providecommand*! erlauben keine neuen Absätze in ihren Argumenten (und
%     erlauben damit eine \enquote{bessere} Fehlerdiagnose, wenn man doch mal
%     \texttt{\}} vergessen hat)
%   \item Das Paket \texttt{xparse} bietet noch umfangreiche Möglichkeiten zur
%     Makrodefinition
%   \end{itemize}
% \end{frame}

% \section{Grundlagen der Textformatierung}

% \begin{frame}[fragile]
%   \frametitle{Quelltextformatierung}

%   \onslide<+->

%   Die Formatierung des Quelltextes ist (wie gesagt) \enquote{fast} beliebig.  Diese
%   Formatierung wird allerdings nicht unbedingt im Dokument wieder gespiegelt.

%   \begin{itemize}
%   \item<+-> Zeilenumbrüche werden (fast) wie Leerzeichen interpretiert:
% \begin{lstlisting}
% Ich bin ein
% Text.
% \end{lstlisting}
%     produziert den gleichen Code wie
% \begin{lstlisting}
% Ich bin ein Text.
% \end{lstlisting}
%   \item<+-> Doppelte Leerzeichen werden wie ein Leerzeichen interpretiert:
% \begin{lstlisting}[showspaces=true]
% Zwei  Leerzeichen
% \end{lstlisting}
% ist das gleiche wie
% \begin{lstlisting}[showspaces=true]
% Zwei Leerzeichen
% \end{lstlisting}
%   \end{itemize}

% \end{frame}

% \begin{frame}[fragile]
%   \frametitle{Absätze, Zeilen- und Seitenumbrüche}

%   \onslide<+->

%   \begin{itemize}
%   \item<+-> Absätze werden durch Leerzeilen oder durch \lstinline{\par} getrennt:

% \begin{lstlisting}
% Ich bin ein erster Absatz.

% Und ich ein zweiter.  \par Und ich ein dritter.
% \end{lstlisting}
%   \item<+-> Zeilenumbrüche mit folgenden Kommandos
%     \begin{itemize}
%     \item<+-> \lstinline{\\} und \lstinline{\newline} erzeugen Zeilenumbruch ohne Ausgleich
%     \item<+-> \lstinline{\linebreak} erzeugt Zeilenumbruch mit Ausgleich
%     \end{itemize}
%   \item<+-> Gleiches mit Seitenumbrüchen
%     \begin{itemize}
%     \item<+-> \lstinline{\newpage} beendet die aktuelle Seite ohne Ausgleich
%     \item<+-> \lstinline{\pagebreak} beendet die aktuelle Seite mit Ausgleich
%     \end{itemize}

%   \end{itemize}

% \end{frame}

% \begin{frame}[fragile]
%   \frametitle{Interpunktion}

%   \onslide<+->

%   Interpunktion (.\,,\,;\,- \dots) in \TeX\ funktioniert meist wie gewohnt, es gibt aber einige
%   Besonderheiten:

%   \begin{itemize}
%   \item<+-> Hinter .\ wird im allgemeinen ein größerer Abstand eingefügt:
% \begin{lstlisting}
% z. B. oder z.\,B.
% \end{lstlisting}
%     wird zu
%     \begin{center}
%       z. B. oder z.\,B.
%     \end{center}
%   \item<+-> \texttt{`\kern0pt`} und \texttt{'\kern0pt'} werden zu englischen
%     Anführungszeichen ``\dots''
%   \item<+-> \texttt{-\kern0pt-} wird zu --, ebenso \texttt{-\kern0pt-\kern0pt-} zu ---
%   \end{itemize}

% \end{frame}

% \section{Textuelles Markup}

% \begin{frame}[fragile]
%   \frametitle{Abschnitte}

%   \begin{itemize}
%   \item<+-> geben die Grobstruktur des Dokuments an
%   \item<+-> In \LaTeX\ mit
%     \begin{itemize}
%     \item \lstinline!\section!, \lstinline!\section*!
%     \item \lstinline!\subsection!, \lstinline!\subsection*!
%     \item \lstinline!\subsubsection!, \lstinline!\subsubsection*!
%     \item \lstinline!\paragraph!, \lstinline!\paragraph*!
%     \item \lstinline!\subparagraph!, \lstinline!\subparagraph*!
%     \end{itemize}
%   \item<+-> *-Formen werden nicht nummeriert und treten auch nicht im
%     Inhaltsverzeichnis auf
%   \item<+-> Je nach Dokumentklasse sind auch noch möglich:
%     \begin{itemize}
%     \item \lstinline!\part!, \lstinline!\part*!
%     \item \lstinline!\chapter!, \lstinline!\chapter*!
%     \end{itemize}
%   \item<+-> Inhaltsverzeichnisse mit
% \begin{lstlisting}
% \tableofcontents
% \end{lstlisting}
%     und zweimaligem Übersetzen.
%   \end{itemize}

% \end{frame}
% \begin{frame}[fragile,fragile]
%   \frametitle{Ausrichtung von Text}

%   \onslide<+->

% \begin{lstlisting}
% \begin{flushleft}
%   Dieser Text ist linksbündig.
% \end{flushleft}
% \end{lstlisting}

%   \onslide<+->

% \begin{lstlisting}
% \begin{flushright}
%   Dieser Text ist rechtsbündig.
% \end{flushright}
% \end{lstlisting}

%   \onslide<+->

% \begin{lstlisting}
% \begin{center}
%   Dieser Text ist zentriert
% \end{center}
% \end{lstlisting}

%   \onslide<+->

% \begin{lstlisting}
% \usepackage{ragged2e}
% \begin{justify}
%   Dieser Text ist im Blocksatz gesetzt.
% \end{justify}
% \end{lstlisting}

% \end{frame}

% \begin{frame}[fragile]
%   \frametitle{\textbf{Fett}, \textit{Kursiv} und Ähnliches}

%   \onslide<+->

%   Für das Markup einzelnen Wörter oder Sätze stehen die folgenden Kommandos zur Verfügung:
%   \bigskip

%   \centering
%   \begin{tabular}[c]{lcl}
%     \lstinline!\textbf{Text}! & $\leadsto$ & \textbf{Text}\\
%     \lstinline!\textsc{Text}! & $\leadsto$ & \textsc{Text}\\
%     \lstinline!\emph{Text}!   & $\leadsto$ & \emph{Text}\\
%     \lstinline!\textsf{Text}! & $\leadsto$ & \textsf{Text}\\
%     \lstinline!\textit{Text}! & $\leadsto$ & \textit{Text}\\
%     \lstinline!\textmd{Text}! & $\leadsto$ & \textmd{Text}\\
%     \lstinline!\textnormal{Text}! & $\leadsto$ & \textnormal{Text}\\
%     \lstinline!\textrm{Text}! & $\leadsto$ & \textrm{Text}\\
%     \lstinline!\textsl{Text}! & $\leadsto$ & \textsl{Text}\\
%     \lstinline!\texttt{Text}! & $\leadsto$ & \texttt{Text}\\
%     \lstinline!\textup{Text}! & $\leadsto$ & \textup{Text}
%   \end{tabular}

% \end{frame}

% \begin{frame}[fragile]
%   \frametitle{Schriftgröße}

%   \begin{onlyenv}<1>
%     \begin{block}{Logisch}
%       \centering
%       \begin{tabular}[c]{cc}
%         \lstinline!\tiny Text!         & \tiny Text \\
%         \lstinline!\scriptsize Text!   & \scriptsize Text \\
%         \lstinline!\footnotesize Text! & \footnotesize Text \\
%         \lstinline!\small Text!        & \small Text \\
%         \lstinline!\normalsize Text!   & \normalsize Text \\
%         \lstinline!\large Text!        & \large Text \\
%         \lstinline!\Large Text!        & \Large Text \\
%         \lstinline!\LARGE Text!        & \LARGE Text \\
%         \lstinline!\huge Text!         & \huge Text \\
%         \lstinline!\Huge Text!         & \Huge Text \\
%       \end{tabular}
%     \end{block}
%   \end{onlyenv}

%   \begin{onlyenv}<2>
%     \begin{block}{Manuell}

% \begin{lstlisting}
% \usepackage{graphicx}
% \scalebox{4}[3]{Beispieltext}
% \resizebox{5cm}{1cm}{Beispieltext 2}
% \end{lstlisting}

%       wird zu

%       \bigskip

%       \scalebox{4}[3]{Beispieltext 1} \resizebox{5cm}{1cm}{Beispieltext 2}

%     \end{block}
%   \end{onlyenv}

% \end{frame}

% \begin{frame}[fragile]
%   \frametitle{Aufzählungen}

%   \LaTeX\ stellt standardmäßig drei Aufzählungstypen zur Verfügung%

%   \onslide<+->

%   \begin{enumerate}
%   \item<+-> \lstinline{itemize} für unnummerierte Aufzählungen
%   \item<+-> \lstinline{enumerate} für nummerierte Aufzählungen
%   \item<+-> \lstinline{description} für Definitionslisten
%   \end{enumerate}

%   \onslide<+->

%   \begin{Beispiel}
%     \begin{columns}
%       \begin{column}{0.5\linewidth}
% \begin{lstlisting}
% \begin{itemize}
% \item Eins
% \item Zwei
% \item Drei
% \end{itemize}
% \end{lstlisting}
%       \end{column}
%       \onslide<+->
%       \begin{column}{0.5\linewidth}
%         \begin{itemize}
%         \item Eins
%         \item Zwei
%         \item Drei
%         \end{itemize}
%       \end{column}
%     \end{columns}
%   \end{Beispiel}

% \end{frame}

% \begin{frame}[fragile]
%   \frametitle{Farben}

%   \onslide<+->

%   Farben werden durch das Paket \lstinline{xcolor} bereit gestellt.

%   \onslide<+->

% \begin{lstlisting}
% \usepackage{xcolor}
% \textcolor{blue}{Blauer Text}
% \textcolor{green}{Grüner Text}
% \textcolor{red!50!blue}{Text blau-rot gemischt}
% \color{gray} Alles, was jetzt noch kommt ist grau
% \end{lstlisting}

%   wird zu\onslide<+->

%   \textcolor{blue}{Blauer Text}
%   \textcolor{green}{Grüner Text}
%   \textcolor{red!50!blue}{Text blau-rot gemischt}
%   \color{gray} Alles, was jetzt noch kommt ist grau

% \end{frame}

% \begin{frame}[fragile]
%   \frametitle{Tabellen}

%   \onslide<+->

%   Sinnvolle Pakete:
% \begin{lstlisting}
% \usepackage{array,booktabs}
% \end{lstlisting}

%   \onslide<+->

%   \bigskip

%   \begin{columns}
%     \begin{column}{0.55\linewidth}
% \begin{lstlisting}
% \begin{tabular}[c]{l|cr}
%   \toprule
%   Tabelle & Kopf  & Kopf  \\
%   \midrule
%   Zeile   & Zelle & Zelle \\
%   Zeile   & Zelle & Zelle \\
%   \bottomrule
% \end{tabular}
% \end{lstlisting}
%     \end{column}
%     \begin{column}{0.45\linewidth}
%       \centering
%       \begin{tabular}[c]{l|cr}
%         \toprule
%         Tabelle & Kopf & Kopf \\
%         \midrule
%         Zeile & Zelle & Zelle \\
%         Zeile & Zelle & Zelle \\
%         \bottomrule
%       \end{tabular}
%     \end{column}
%   \end{columns}

% \end{frame}

% \section{Bilder einbinden und Grafiken erstellen}

% \begin{frame}[fragile]
%   \frametitle{Bilder einbinden}

%   \onslide<+->

%   \begin{itemize}
%   \item Einbinden von Graphiken in \LaTeX\ mit Hilfe des Pakets \texttt{graphicx}
%   \item Befehl
% \begin{lstlisting}
% \includegraphics[(*@\textit{Optionen}@*)]{(*@\textit{Bildname}@*)}
% \end{lstlisting}
%   \end{itemize}

%   \onslide<+->

%   \begin{Beispiel}
% \begin{lstlisting}
% \centerline{\includegraphics[width=0.3\linewidth,
%   keepaspectratio]{meinBild.pdf}}
% \end{lstlisting}

%     ergibt

%     \centerline{\includegraphics[width=0.3\linewidth,keepaspectratio]{pics/meinBild.pdf}}
%   \end{Beispiel}

% \end{frame}

% \begin{frame}[fragile]
%   \frametitle{Optionen zum Einbinden von Graphiken}

%   \onslide<+->

%   Meist verwendete Optionen von \lstinline{\includegraphics} sind
%   \begin{itemize}
%     \item \texttt{width}, \texttt{height} für Breite und Höhe
%     \item \texttt{keepaspectratio}, so dass nach Angabe von Breite oder Höhe
%       das Bild automatisch skaliert wird
%     \item \texttt{scale} zur Skalierung des Bildes
%     \item \texttt{angle} zur Angabe eines Drehwinkels
%     \item \texttt{origin} zur Angabe des Drehpunktes
%   \end{itemize}

%   \onslide<+->

%   \begin{Beispiel}
% \begin{lstlisting}
% \centerline{\includegraphics[scale=1.2,origin=cc,
%     angle=42]{bild.jpg}}
% \end{lstlisting}

%     \centerline{\includegraphics[scale=1.2,origin=cc,angle=42]{pics/bild.jpg}}
%   \end{Beispiel}

% \end{frame}

% \begin{frame}[fragile]
%   \frametitle{Weitere Befehle aus \texttt{graphicx}}

%   \onslide<+->

%   \begin{itemize}[<+->]
%   \item Drehen von Inhalten mit
%     \lstinline!\rotatebox[$\textit{Optionen}$]{$\textit{Winkel}$}{$\textit{Inhalt}$}!
%     \onslide<+->
% \begin{lstlisting}
% \rotatebox[origin=lB]{-30}{TextTextTextText}
% \end{lstlisting}
%     \rotatebox[origin=lB]{-30}{TextTextTextText}
%   \item \lstinline!\resizebox{$\textit{Breite}$}{$\textit{Höhe}$}{$\textit{Text}$}!
%     \onslide<+->
% \begin{lstlisting}
% \resizebox{1cm}{.4cm}{Hier ist es eng...}
% \end{lstlisting}
%     \resizebox{1cm}{.4cm}{Hier ist es eng...}
%   \item \lstinline!\scalebox{$\textit{horizontal}$}[$\textit{vertikal}$]{$\textit{Text}$}!
%     \onslide<+->
% \begin{lstlisting}
% \scalebox{3}[-1]{Breitergehtnicht}
% \end{lstlisting}
%     \scalebox{3}[-1]{Breitergehtnicht}
%   \end{itemize}
% \end{frame}

% \begin{frame}[fragile]
%   \frametitle{Gleitumgebungen}

%   \onslide<+->

%   Größere Bilder werden mittels \emph{Gleitumgebungen} gesetzt:
% \begin{lstlisting}
% \begin{figure}[$\textit{Optionen}$]
%   $\dots$
%   \caption{Bildunterschrift}
% \end{figure}
% \end{lstlisting}
%   \LaTeX\ platziert dann die Bilder auf der aktuellen oder auf einer der folgenden Seiten.
%   \onslide<+-> Die Platzierung wird durch die entsprechenden \textit{Optionen} angegeben:
%   \begin{description}
%   \item[h] Platzierung an der aktuellen Stelle
%   \item[t] Platzierung oben auf einer Seite
%   \item[b] Platzierung unten auf einer Seite
%   \item[p] Platzierung auf einer extra Seite
%   \end{description}

%   \onslide<+->

%   Optionen können gemischt werden.

%   \onslide<+->

%   Für Tabellen gibt es die spezielle \texttt{table}-Umgebung.

% \end{frame}

% \begin{frame}[fragile]
%   \frametitle{Grafiken erstellen}

%   \begin{center}
%     \begin{tikzpicture}[
%         thick,
%         >=stealth',
%         dot/.style = {
%           draw,
%           fill = white,
%           circle,
%           inner sep = 0pt,
%           minimum size = 4pt
%         }
%       ]
%       \coordinate (O) at (0,0);
%       \draw[->] (-0.3,0) -- (8,0) coordinate[label = {below:$x$}] (xmax);
%       \draw[->] (0,-0.3) -- (0,5) coordinate[label = {right:$f(x)$}] (ymax);
%       \path[name path=x] (0.3,0.5) -- (6.7,4.7);
%       \path[name path=y] plot[smooth] coordinates {(-0.3,2) (2,1.5) (4,2.8) (6,5)};
%       \scope[name intersections = {of = x and y, name = i}]
%         \fill[gray!20] (i-1) -- (i-2 |- i-1) -- (i-2) -- cycle;
%         \draw      (0.3,0.5) -- (6.7,4.7) node[pos=0.8, below right] {Sekante};
%         \draw[red] plot[smooth] coordinates {(-0.3,2) (2,1.5) (4,2.8) (6,5)};
%         \draw (i-1) node[dot, label = {above:$P$}] (i-1) {} -- node[left,yshift=-3pt]
%           {$f(x_0)$} (i-1 |- O) node[dot, label = {below:$x_0$}] {};
%         \path (i-2) node[dot, label = {above:$Q$}] (i-2) {} -- (i-2 |- i-1)
%           node[dot] (i-12) {};
%         \draw           (i-12) -- (i-12 |- O) node[dot,
%                                   label = {below:$x_0 + \varepsilon$}] {};
%         \draw[blue, <->] (i-2) -- node[right] {$f(x_0 + \varepsilon) - f(x_0)$}
%                                   (i-12);
%         \draw[blue, <->] (i-1) -- node[below] {$\varepsilon$} (i-12);
%         \path       (i-1 |- O) -- node[below] {$\varepsilon$} (i-2 |- O);
%         \draw[gray]      (i-2) -- (i-2 -| xmax);
%         \draw[gray, <->] ([xshift = -0.5cm]i-2 -| xmax) -- node[fill = white]
%           {$f(x_0 + \varepsilon)$}  ([xshift = -0.5cm]xmax);
%       \endscope
%     \end{tikzpicture}
%   \end{center}

%   \onslide<2->{mit \textcolor{red}{Ti\textit{k}Z} $\leadsto$ später!}

%   \vfill\hbox{}\hfill\hbox{\tiny\url{http://www.texample.net/tikz/examples/linear-regression/}}

% \end{frame}

% \section{Mathematik}

% \begin{frame}
%   \frametitle{Wozu \TeX\ geschaffen wurde...}

%   \onslide<+->

%   \begin{equation*}
%     \frac{f\left(\zeta\right)}{\zeta-z_0} = \frac{f\left(\zeta\right)}
%     {\zeta-z_0}\frac{1}{
%       1-\frac{z-z_0}{\zeta-z_0}} = \sum_{n=0}^{\infty}\frac{f\left(\zeta\right)}
%     {\zeta-z_0}
%     \left(\frac{z-z_0}{\zeta-z_0}\right)^n
%   \end{equation*}

%   \medskip

%   \begin{equation*}
%     0 \neq \left|\, \frac{1}{10^{10}} \left( \sum_{n = -\infty}^{\infty}
%         e^{\frac{n^2}{10^{10}}} \right)^2 - \pi \,\right|
%     \le 10^{-42 \cdot 10^9}
%   \end{equation*}

%   \medskip

%   \begin{equation*}
%     \frac{1}{\pi} = \frac{2\sqrt{2}}{9801} \sum^\infty_{k=0} \frac{(4k)!(1103+26390k)}{(k!)^4 396^{4k}}
%   \end{equation*}

% \end{frame}

% \begin{frame}
%   \frametitle{Setzen mathematischer Formeln}

%   \onslide<+->

%   Prinzipiell ist Mathematik eine ganz andere Welt (nicht nur in \LaTeX), denn
%   \begin{itemize}
%   \item<+-> das Setzen mathematischer Formeln erfolgt in einer eigenen Umgebung
%   \item<+-> mit eigenen Befehlen,
%   \item<+-> eigener Schrift,
%   \item<+-> eigenen Einstellungen
%   \item<+-> und eigenen Tücken...
%   \end{itemize}

%   \onslide<+->

%   aber es ist einfacher (und schöner) als in den meisten (allen) anderen Textsatzsystemen
% \end{frame}

% \begin{frame}[fragile,fragile]
%   \frametitle{Das Grundgerüst}

%   \onslide<+->

%   2 Möglichkeiten, Formeln zu setzen:
%   \begin{itemize}
%   \item<+-> Im laufenden Text mit \lstinline!$\text{\$}\dots\text{\$}$! oder
%     \lstinline!\($\dots$\)!
%   \item<+-> oder abgesetzt mit \lstinline!\[$\dots$\]! oder
%     \lstinline!\begin{equation*}$\dots$\end{equation*}!
%   \end{itemize}

%   \onslide<+->

%   Dies ist eine $\sum_{i=0}^\infty\frac{1}{n}=\infty$ Textformel und dies ist eine
%   \begin{equation*}
%     \sum_{i=0}^\infty\frac{1}{n} = \infty
%   \end{equation*}
%   abgesetzte Formel.

%   \onslide<+->

% \begin{lstlisting}
% Dies ist eine $\text{\$}$\sum_{i=0}^\infty\frac{1}{n}=\infty$\text{\$}$
% Textformel und dies ist eine
% \begin{equation*}
%   \sum_{i=0}^\infty\frac{1}{n} = \infty
% \end{equation*}
% abgesetzte Formel.
% \end{lstlisting}
% \end{frame}

% \begin{frame}[fragile]
%   \frametitle{Grundlegende Formelelemente}

%   \onslide<+->

%   Einige grundlegende Formelelemente sind
%   \begin{itemize}
%   \item Buchstaben, dargestellt als \textit{jeweils ein} Symbol, $xyz$,
%   \item Zahlen, einfach gesetzt $123$,
%   \item griechische Buchstaben wie $\gamma, \varepsilon, \xi, \ldots$, und hebräische wie
%     $\aleph, \beth, \ldots$
%   \item Operationen wie $+$,$-$ und $\cdot$ (mit \lstinline!\cdot!)
%   \item Exponenten, gesetzt mit \textasciicircum, z.B. $\text{\lstinline!x^2!}
%     \mathrel{\hat=} x^2$,
%   \item Indizes mit \_, z.B. $\text{\lstinline!x_2!} \mathrel{\hat=} x_2$,
%   \item \lstinline!\frac{$\textit{Zähler}$}{$\textit{Nenner}$}! für $\frac{1}{n}$,
%   \item Wurzeln mit \lstinline!\sqrt[$\textit{Grad}$]{$\textit{Radikant}$}!, z.B.
%     \lstinline!\sqrt[3]{x}! ${} \mathrel{\hat=} \sqrt[3]{x}$
%   \end{itemize}

%   \onslide<+->

%   und natürlich kann man das \textit{alles} auch kombinieren:
%   \begin{equation*}
%     \gamma + \frac{\aleph^{\beth - \frac{1+\beta\cdot x^2}{\sqrt{2}}}-5}{\sqrt[2+\sqrt{\phi}]
%       {42-\alpha-\theta-\mu}^\pi}
%   \end{equation*}

% \end{frame}

% \begin{frame}[fragile]
%   \frametitle{Klammern}

%   \onslide<+->

%   \LaTeX\ unterstützt jegliche Formen von Klammern:
%   \begin{equation*}
%     (x), \{x\},[x],\lfloor x\rfloor, \lceil x\rceil,|x|,\langle x \rangle,\ldots
%   \end{equation*}

%   \onslide<+->

%   Größenanpassung mit \lstinline{\left} und \lstinline{\right}, jeweils paarig:

% \begin{lstlisting}
% \left(\sum_{i=0}^1 5 = 10\right)
% \end{lstlisting}

%   ergibt
%   \vspace*{-3ex}

%   \begin{equation*}
%     \left(\sum_{i=0}^1 5 = 10\right)
%   \end{equation*}

%   \vspace*{1ex}

%   \onslide<+->

% \begin{lstlisting}
% \left|\sum_{i=0}^1 5 = 10\right[
% \end{lstlisting}

%   ergibt
%   \vspace*{-3ex}

%   \begin{equation*}
%     \left|\sum_{i=0}^1 5 = 10\right[
%   \end{equation*}

% \end{frame}

% \begin{frame}[fragile]
%   \frametitle{Klammern II}

%   \onslide<+->

%   oder auch

% \begin{lstlisting}
% \int_0^1x^2 \mathop{} \mathsf d x =
%   \left.\frac{1}{3}x^3\right|_0^1
% \end{lstlisting}
%   \begin{equation*}
%     \int_0^1x^2 \mathop{} \mathsf d x = \left.\frac{1}{3}x^3\right|_0^1
%   \end{equation*}

%   denn \verb|.| ist das Sonderzeichen für eine leere Klammer.

%   \onslide<+->
%   \medskip

%   Klammern gibt es auch über- und unterhalb von Formeln mit Hilfe von
%   \lstinline{\underbrace} und \lstinline{\overbrace}:
% \begin{lstlisting}
% 1+\underbrace{ 2 + \overbrace{3 + 4}^{7} }_{9} = 10
% \end{lstlisting}
%   \begin{equation*}
%     1+\underbrace{ 2 + \overbrace{3 + 4}^{7} }_{9} = 10
%   \end{equation*}

% \end{frame}

% \begin{frame}[fragile]
%   \frametitle{Grenzen}

%   \onslide<+->

%   Änderbar mit den Befehlen \lstinline{\limits} und \lstinline{\nolimits}, also z.B.

% \begin{lstlisting}
% \sum\nolimits_{i=1}^\infty\frac{1}{n^2}
%    = \frac{\pi^2}{6}
% \end{lstlisting}

%   \begin{equation*}
%     \sum\nolimits_{i=1}^\infty\frac{1}{n^2} = \frac{\pi^2}{6}
%   \end{equation*}

%   \bigskip

%   \onslide<+->

% \begin{lstlisting}
% \int\limits_0^1x^2 \mathop{} \mathsf d x
%    = \frac{1}{3}
% \end{lstlisting}

%   \begin{equation*}
%     \int\limits_0^1 x^2 \mathop{} \mathsf d x = \frac{1}{3}
%   \end{equation*}

% \end{frame}

% \begin{frame}[fragile]
%   \frametitle{Funktionen}

%   \onslide<+->

%   \begin{block}{Beobachtung}
%     \vspace*{-\baselineskip}
%     \begin{equation*}
%       sin(x) \neq \sin(x)
%     \end{equation*}
%   \end{block}

%   \onslide<+->

%   daher: allgemein gebräuchliche Funktionen werden gesondert behandelt:

%   \smallskip

%   \begin{center}
%     \begin{tabular}{ll|ll|ll|ll}
%       \toprule
%       \lstinline{\log} & $\log$ & \lstinline{\lg} & $\lg$ & \lstinline{\ln} & $\ln$ & \lstinline{\lim} & $\lim$ \\
%       \lstinline{\sin} & $\sin$ & \lstinline{\arcsin} & $\arcsin$ & \lstinline{\sinh} & $\sinh$ & \lstinline{\cos} & $\cos$ \\
%       \lstinline{\arccos} & $\arccos$ & \lstinline{\cosh} & $\cosh$ & \lstinline{\tan} & $\tan$ & \lstinline{\tanh} & $\tanh$ \\
%       \lstinline{\arctan} & $\arctan$ & \lstinline{\cot} & $\cot$ & \lstinline{\coth} & $\coth$ & \lstinline{\max} & $\max$ \\
%       \lstinline{\min} & $\min$  & \lstinline{\arg} & $\arg$ & \lstinline{\det} & $\det$  & \lstinline{\Pr} & $\Pr$ \\
%       \bottomrule
%     \end{tabular}
%   \end{center}

%   \smallskip

%   \onslide<+->

%   Damit sind auch korrekte Indizierungen möglich:
% \begin{lstlisting}
% lim_{n\to\infty} \frac{1}{n} = 0 \quad
%   \not\equiv \quad \lim_{n\to\infty}\frac{1}{n}=0
% \end{lstlisting}
%   \begin{equation*}
%     lim_{n\to\infty} \frac{1}{n} = 0 \quad \not\equiv \quad \lim_{n\to\infty}\frac{1}{n}=0
%   \end{equation*}
% \end{frame}

% \begin{frame}[fragile]
%   \frametitle{Mengen}

%   \onslide<+->

%   Mengenbedingungen werden mit dem Kommando \lstinline{\mid} gesetzt:
% \begin{lstlisting}
% \{\, n \in \mathbb N \mid n \ge \pi \,\}
% \end{lstlisting}
%   \vskip1ex
%   \begin{equation*}
%     \{\, n \in \mathbb N \mid n \ge \pi \,\}
%   \end{equation*}

%   \onslide<+->

%   Ein einfacher Strich | reicht nicht aus!
%   \begin{equation*}
%     \{\, n \in \mathbb N | n \ge \pi \,\}
%   \end{equation*}

% \end{frame}

% \begin{frame}[fragile]
%   \frametitle{Akzente}

%   \onslide<+->

%   Auch Akzente sind im Mathematikmodus neu:

%   \medskip

%   \begin{center}
%     \begin{tabular}{ll|ll|ll}
%       \toprule
%       \lstinline{\acute}    & $\acute x$    & \lstinline{\hat}       & $\hat x$          & \lstinline{\grave}   & $\grave x$ \\
%       \lstinline{\ddot}     & $\ddot x$     & \lstinline{\tilde}     & $\tilde x$        & \lstinline{\bar}     & $\bar x$ \\
%       \lstinline{\breve}    & $\breve x$    & \lstinline{\check}     & $\check x$        & \lstinline{\dot}     & $\dot x$ \\
%       \lstinline{\vec}      & $\vec x$      & \lstinline{\widetilde} & $\widetilde{xyz}$ & \lstinline{\widehat} & $\widehat{xyz}$ \\
%       \lstinline{\mathring} & $\mathring x$ & \lstinline{\prime}     & $x^{\prime}$        & \lstinline{'}        & $x'$\\
%       \bottomrule
%     \end{tabular}
%   \end{center}

%   \medskip

%   und mit Hilfe einiger Zusatzpakete gibt es noch viel mehr\ldots
% \end{frame}

% \begin{frame}[fragile]
%   \frametitle{und noch viel mehr Symbole}

%   \onslide<+->

%   Noch viel mehr Symbole:

%   \begin{center}
%     \hbox{\url{http://tug.ctan.org/info/symbols/comprehensive/symbols-a4.pdf}}
%   \end{center}

%   \vspace*{-\baselineskip}

%   \onslide<+->

%   und noch viel mehr zu sagen, z.B.
%   \begin{itemize}
%   \item<+-> gibt es 8 unterschiedliche Symbolklassen
%   \item<+-> eigentlich 4 (8) Modi im Mathematikmodus
%   \item<+-> Text in mathematischen Formeln (mit \lstinline!\text{ein wenig Text}!)
%   \item<+-> manuelle Feinjustierung, z.B.
%     \begin{equation*}
%       \int_0^1x^2dx \qquad \int_0^1 x^2 \mathop{} \mathrm{d}x
%     \end{equation*}
%   \item<+-> \ldots
%   \end{itemize}

% \end{frame}

% \begin{frame}[fragile]
%   \frametitle{Referenzen auf Gleichungen}

%   \onslide<+->

%   Verweise auf Formeln mit

% \begin{lstlisting}
% \begin{equation}\label{$\textit{label}$}
%   $\dots$
% \end{equation}
% \end{lstlisting}

%   z.B.

% \begin{lstlisting}
% \begin{equation}\label{eqn:hahah}
%   2 + 2 = 5
% \end{equation}
% und die Formel~(\ref{eqn:hahah}) ist doch richtig!
% \end{lstlisting}

%   wird zu:
%   \begin{equation}\label{eqn:hahah}
%     2 + 2 = 5
%   \end{equation}
%   und die Formel~(\ref{eqn:hahah}) ist doch richtig!

%   \onslide<+->

%   Hinweis: \texttt{equation*} statt \texttt{equation} unterdrückt die Nummerierung.

% \end{frame}

% \begin{frame}[fragile]
%   \frametitle{Umgebungen}

%   \onslide<+->

%   Ausgerichtete Formeln mit
% \begin{lstlisting}
% \usepackage{amsmath}
% \begin{align*}
%   $\dots$
% \end{align*}
% \end{lstlisting}

%   \onslide<+->

%   z.B.

% \begin{lstlisting}
% \begin{align*}
%   2x + y + 5x + z
%   &= 2x + 5x + y + z \\
%   &= 7x + y + z
% \end{align*}
% \end{lstlisting}

%   wird zu\vspace*{-\baselineskip}
%   \begin{align*}
%     2x + y + 5x + z &= 2x + 5x + y + z \\
%                     &= 7 x + y + z
%   \end{align*}

% \end{frame}

% \begin{frame}[fragile]
%   \frametitle{Umgebungen II -- \AmS\TeX}

%   \onslide<+->

%   Das Paket \texttt{amsmath} stellt weiter Umgebungen bereit:
%   \begin{itemize}
%   \item \texttt{split} für den Einsatz mehrzeiliger Formeln in der \texttt{equation}
%   \item \texttt{align}
%   \item \texttt{multline} für \enquote{zu lange} Formeln
%   \item \texttt{gather} für lose zusammengeworfenen Formeln
%   \item ...
%   \end{itemize}

%   und \AmS\TeX\ erlaubt den Umbruch mehrseitiger Formeln mit \lstinline!\allowdisplaybreaks!.

% \end{frame}

% \begin{frame}[fragile]
%   \frametitle{Tabellen (Matrizen)}

%   \onslide<+->

%   Für Matrizen gibt es die Umgebung \texttt{pmatrix}:

% \begin{lstlisting}
% \begin{pmatrix}
%   1 & 2 & 3 \\
%   4 & 5 & 6 \\
%   7 & 8 & 9
% \end{pmatrix}
% \end{lstlisting}
%   \begin{equation*}
%     \begin{pmatrix}
%       1 & 2 & 3 \\
%       4 & 5 & 6 \\
%       7 & 8 & 9
%     \end{pmatrix}
%   \end{equation*}
%   ist aber auch mit der Umgebung \texttt{array} möglich.

% \end{frame}


% \section{Referenzieren}

% \begin{frame}[fragile]
%   \frametitle{Label und Referenzen}

%   \onslide<+->

%   \LaTeX\ erlaubt die automatische Handhabung von Verweisen mittels \lstinline{\label} und
%   \lstinline{\ref}.

%   \onslide<+->

% \begin{lstlisting}
% \section{Abschnitt}
% \label{abschnittslabel}

% \begin{enumerate}
% \item\label{item:1} Eintrag
% \end{enumerate}

% \begin{figure}
%   $\dots$
%   \caption{\label{figure} Bildunterschrift}
% \end{figure}
% \end{lstlisting}

%   \onslide<+->

%   Verweis mittels \lstinline!\ref{figure}! im Text.

% \end{frame}

% \begin{frame}[fragile]
%   \frametitle{Literaturverweise}

%   \onslide<+->

%   Verweise auf Literatur werden mittels \lstinline{cite} erreicht:

% \begin{lstlisting}
% \cite{MyPaper}
% $\dots$
% \begin{thebiliography}[MyPaper]
% \bibitem{BHM14}{MyPaper} \LaTeX, Linux, Python
% $\dots$
% \end{thebibliography}
% \end{lstlisting}

%   \onslide<+->

%   Einfacher mittels \emph{externer} Literaturangaben und Bib\TeX.

% \end{frame}

% \begin{frame}[fragile,fragile]
%   \frametitle{Bib\TeX}

%   \onslide<+->

%   Literaturangabe in \emph{externer Datei} (z.B. \texttt{my.bib})

% \begin{lstlisting}
% @article{MyPaper,
%   title = {\LaTeX, Linux, Python},
%   author = {Daniel Borchmann and Tom Hanika
%     and Maximilian Marx},
%   journal = {Advancing Technical Understanding of
%     Mathematics Students},
%   year = {2014},
% }
% \end{lstlisting}

%   \onslide<+->

%   Einbinden im Dokument mit zugehörigem Stil
% \begin{lstlisting}
% \bibliography{my}
% \bibliographystyle{plain}
% \end{lstlisting}

%   \onslide<+->

%   Nach dem ersten Durchlauf von \texttt{pdflatex} (o.ä.) Aufruf von \texttt{bibtex}, und
%   danach nochmal Durchlauf von \texttt{pdflatex}.

% \end{frame}

% \begin{frame}[fragile]
%   \frametitle{Bib\LaTeX}

%   \onslide<+->

%   Problem mit Bib\TeX: Stile schwer anpassbar.

%   \onslide<+->

%   Ausweg: Bib\LaTeX\ und Biber

%   \onslide<+->

% \begin{lstlisting}
% \usepackage[style=numeric-comp,
%   backref=true, isbn=true,
%   doi=false, maxbibnames=5]
%   {biblatex}
% \addbibresource{my.bib}

% \begin{document}

% \printbibliography{}

% \end{document}
% \end{lstlisting}

%   \onslide<+->

%   Aufruf von \texttt{biber} statt \texttt{bibtex}.

% \end{frame}

% \begin{frame}[fragile]
%   \frametitle{Der Index}

%   \onslide<+->

%   Um einen Index zu erstellen geht man wie folgt vor:
%   \begin{itemize}
%   \item Einbinden des Paktes \textit{makeidx} mit \lstinline!\usepackage{makeidx}!
%   \item Aufruf des Befehls \lstinline{\makeindex} \textit{in der Präambel}
%   \item Aufruf von \lstinline!\printindex! an der Stelle, an der der Index gedruckt werden
%     soll
%   \end{itemize}

%   \onslide<+->

%   \begin{itemize}
%   \item Nach dem ersten \LaTeX-Durchlauf sollte das Programm \texttt{makeindex}
%     auf der erzeugten *.idx-Datei aufgerufen werden, um den Index zu erstellen
%   \item Ein weiterer \LaTeX-Durchlauf erzeugt dann das gewünschte Dokument
%   \end{itemize}

%   \onslide<+->

%   \begin{itemize}
%   \item Indexeinträgen mit \lstinline!\index{Eintrag}!
%   \item Untereinträge mit \lstinline?\index{Haupteintrag!Untereintrag}?
%   \item Viele weitere Formatierungsmöglichkeiten
%   \end{itemize}

% \end{frame}

\end{document}
%%% Local Variables:
%%% mode: latex
%%% TeX-master: t
%%% TeX-engine: luatex
%%% ispell-local-dictionary: "de_DE"
%%% End:

%  LocalWords:  Textuelles Gliederungsstufe scrartcl scrreprt scrbook Script microtype
%  LocalWords:  Mikrotypographie geometry fontenc inputenc babel enumitem array booktabs
%  LocalWords:  listings hyperref amsmath amssymb mathtools ntheorem fragile
%  LocalWords:  Quelltextformatierung Seitenumbrüche Aufzählungstypen Formelelemente
