% Dokumentenklasse
% immer am Anfang des Dokuments
\documentclass{scrartcl}

% Präambel
\usepackage[utf8]{inputenc} % Eingabekodierung
\usepackage[T1]{fontenc}    % Schrifkodierung
\usepackage[ngerman]{babel} % optimiert für Deutsch (z.B. Titel, Silbentrennung)

\title{Mein erstes \LaTeX-Dokument}
\author{LLP-Kurs}
\date{nach Ostern}

\newcommand{\wichtig}[1]{(Wichtig!) \emph{#1} (Achtung, dieses Wort sollte man sich merken!)}
\newcommand{\tb}{\textbackslash}
\newcommand{\unwichtig}[2][Tschau]{(unwichtig: #1, #2)}

% Ende Präambel
% Beginn eigentliches Dokument
\begin{document}
\maketitle
  Hallo. Jetzt schreibe ich ein deutschen Text mit Umlauten: ä, ö, ü und Accenten: é ù %ȩ

  Das eigentliche Dokument wird nun zwischen
  \tb begin\{document\} und \tb end\{document\}
  reingeschrieben. \TeX \wichtig{formatiert}
  es dann \wichtig{ansprechend}.

  Bei Fülltexten sollte man drauf achten, dass es nicht
  allzu viel Sinn macht, lange darüber nachzudenken, was
  man wie schreibt.

  Ich kann auch Makros ineinander tun: Jetzt kommt ein echt
  wichtiges Wort:
  \emph{\wichtig{Genialkohöllischer Wunschpunsch}}

  \unwichtig[tschüss]{Hallo}
\end{document}
